\begin{document}

Recall that our goal was to analyze the factors which made a date successful.

Comparing the results obtained with the basic statistical and descriptive analysis and the PCA, we can affirm that they are coherent up to a point. In both of them we arrived to the following results:
\begin{itemize}
    \item Intelligence is NOT a nice feature to have if your goal is to have a match.
    \item In contrast, being fun and attractive gives you many changes of getting a match.
    \item Women are more likely to get a match. Men say yes to many women, but the opposite does not hold.
\end{itemize}

However, with bivariate analysis we observed additional results than we could not appreciate in the PCA: people with a career related to maths are less likely to get a match, and race is not quite important to have a match. The opposite held, too: in PCA we observed that common interests were related to having a match, and this information could not be extracted from the bivariate analysis. Also, while doing PCA we observed income does not seem to be quite related to any other variable apart from age.

As far as the clustering and profiling are concerned, we have found that the difference of age is a relevant factor to get a match (the greater the difference, the less matches people get). The information about the difference of age is not necessarily contradictory with the results obtained with the bivariate and factorial analysis, but with the latter methods we had not found them. However, for this conclusion and the following ones, again, we have to take into account that the p-values are low. 

What we have found to be coherent with the results obtained by both bivariate and PC analysis is that the marks obtained to access the university do not seem to be a relevant factor as far as dating is concerned. However, in profiling we have seen that income is a relevant factor, but the difference might not be significant. Those subjects who go out several times a week have a higher chance to end the date up in a match. We have found other relations stated in the profiling section, too.

Neither of the three methods have found the race to be a relevant factor.

\end{document}
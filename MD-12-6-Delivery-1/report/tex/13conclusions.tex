\begin{document}
In order to sum up this project we can outline the following conclusions:
\begin{itemize}
    \item We consider that we have successfully applied to real world data the data mining techniques (pre-processing, basis statistical and descriptive analysis, PCA, hierarchical clustering and profiling) which we have been learning during this course.
    \item However, we have learned the hard way that in practice data mining is much more difficult than in theory.
    \item That is to say, our dataset (and many other real world datasets, we assume) has proven to be particularly difficult to work with. Many techniques cannot be applied directly and you have to adapt them to your particular case.
    \item Still, we consider that we have extracted some useful information (knowledge discovery) regarding the factors which make dates successful. We have obtained a similar answer to the initial question by using different techniques applied by independent teams (different tasks were assigned to different members of our group).
    \item We can affirm that being (or at least being considered) funny and attractive increases the chances of having a match. On the contrary, being perceived as intelligent appears not to be a good idea if the goal is to be liked by the other sex. We have found a weak relation between having common interests and getting a match. Also, women get more matches than men (so matches are more evenly distributed among women).
    \item The differences between clusters found in the profiling step apparently were not significant enough (the p-values were low). However, as far as we know social sciences tend to find low p-values in general, so it is an issue but we think it is a very common issue in this field.
    \item As a suggestion to improve this project, with perspective, we believe that perhaps it would have been a better idea to further semantically collapse the dataset in order to have still less variables but more meaningful.
    \item As far as the team work, working plan and planning are concerned, we all agree that we have achieved our goals and everybody has contributed to the team by doing the assigned tasks. Nevertheless, the final working plan and planning differs from the original one. The main drawback we have had is the fact that pre-processing took so much longer than expected. Also, managing a team of as many as 6 members has proven to be a challenge (many different ideas, many different timetables...). This point will be detailed in the next section.
    \item As far as the team work, working plan and planning are concerned, we all agree that we have achieved our goals and everybody has contributed to the team by doing the assigned tasks. Nevertheless, the final working plan and planning differs from the original one. The main drawback we have had is the fact that pre-processing took so much longer than expected. Also, managing a team of as many as 6 members has proven to be a challenge (many different ideas, many different timetables...). This point will be detailed in the next section.
    \item As a future work, we suggest studying other aspects included in this dataset that were out of the scope of this project, basically analyzing the difference between self-perceptions and actual perceptions by other people.
\end{itemize}
\end{document}
\begin{document}
Data mining is the process of sorting through large data sets to identify patterns and establish relationships to solve problems through data analysis. Data mining tools allow enterprises to predict future trends. 

In data mining, association rules are created by analyzing data for frequent if/then patterns, then using the support and confidence criteria to locate the most important relationships within the data. Support is how frequently the items appear in the database, while confidence is the number of times if/then statements are accurate.

Other data mining parameters include Sequence or Path Analysis, Classification, Clustering and Forecasting. Sequence or Path Analysis parameters look for patterns where one event leads to another later event. A Sequence is an ordered list of sets of items, and it is a common type of data structure found in many databases. A Classification parameter looks for new patterns, and might result in a change in the way the data is organized. Classification algorithms predict variables based on other factors within the database.

Clustering parameters find and visually document groups of facts that were previously unknown. Clustering groups a set of objects and aggregates them based on how similar they are to each other.

There are different ways a user can implement the cluster, which differentiate between each clustering model. Fostering parameters within data mining can discover patterns in data that can lead to reasonable predictions about the future, also known as predictive analysis.

During this project, we are going to make use of the methods previously mentioned to study a data set we chose.
\end{document}
\begin{document}
The motivation of this project is to apply the concepts we have been studying in the Data Mining subject in order to extract knowledge from a dataset based on a survey taken to participants of a series of speed dating. Thus, we will both learn data mining concepts by applying them in real data and hopefully discover what makes a date successful.

In particular, we will apply the following concepts which we have studied in the course (among others):
\begin{itemize}
    \item Pre-processing.
    \item Basic statistical and descriptive analysis.
    \item Factorial analysis.
    \item Hierarchical clustering.
    \item Profiling.
    \item The R programming language.
\end{itemize}

As far as the dataset is concerned, our intention is to analyze what are the most important features on a date for being successful, taking into account the information available in the survey. For instance, we would like to discover whether the age or the difference of ages have a big impact or otherwise it is more important to have a high income or having a certain career and income.

We think that this dataset can be very interesting to analyze because there are different points that we can study. The main question that we want to answer is: What attributes influence the selection of a romantic partner?

Nevertheless, we have to say that in order to be able to finish this project in time without compromising its quality, we have had to narrow the scope of our data mining process. That is to say, not all the questions in the survey have been taken into account (as detailed in the pre-processing chapter). For instance, there were questions which were actually not related on the dates itself, but on the perceptions that individuals had of themselves. This information could lead to another whole data mining project with the aim of comparing the perceptions which individuals have of themselves with the way other people actually perceive them, but it is clearly out of the scope of this project.

This dataset has proven to be not very \textit{data-science-friendly}, although we were aware of this concern from the beginning of the project. We chose it not because it was easy to work with, but because it was interesting and we were keen on the challenge. We hope that our effort will have been worth it.
\end{document}